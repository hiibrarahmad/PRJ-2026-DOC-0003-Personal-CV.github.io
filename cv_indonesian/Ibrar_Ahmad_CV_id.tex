% !TeX program = xelatex
% !TeX encoding = UTF-8
\documentclass[11pt,a4paper]{awesome-cv}
\usepackage{fontspec}

% Color for highlights
\colorlet{awesome}{awesome-red}

% Set boolean value for photo/quote/etc
\setbool{acvSectionColorHighlight}{true}

% If you would like to change the social information separator from a pipe (|) to something else
\renewcommand{\acvHeaderSocialSep}{\quad\textbar\quad}

% Project list formatting: consistent spacing and alignment
\newenvironment{cvprojitems}{%
  \begin{justify}
  \begin{itemize}[leftmargin=2ex, itemsep=0.4ex, topsep=0.3ex, parsep=0pt, partopsep=0pt]
    \setlength{\parskip}{0pt}
    \renewcommand{\labelitemi}{\bullet}
}{%
  \end{itemize}
  \end{justify}
}

%-------------------------------------------------------------------------------
% PERSONAL INFORMATION
%-------------------------------------------------------------------------------
\name{Ibrar}{Ahmad}
\position{Insinyur Sistem Tertanam | Perangkat Keras EEG \& Biosensor}
\address{Peshawar, 25000, Pakistan}

\mobile{(+92) 309-8767-341}
\email{hiibrarahmad@gmail.com}
\homepage{https://hiibrarahmad.github.io}
\github{hiibrarahmad}
\linkedin{hiibrarahmad}

%-------------------------------------------------------------------------------
\begin{document}

% Print the header
\makecvheader

% Print the footer with page numbers
\makecvfooter
  {\today}
  {Ibrar Ahmad~~~\textperiodcentered~~~Daftar Riwayat Hidup}
  {\thepage}

%-------------------------------------------------------------------------------
% SECTION: Summary
%-------------------------------------------------------------------------------
\cvsection{Ringkasan Profesional}

\begin{cvparagraph}
Insinyur perangkat keras tertanam dengan spesialisasi pada desain perangkat keras biosensor/EEG, desain PCB berkecepatan tinggi, integritas sinyal \& daya, pengembangan firmware (C/C++, FreeRTOS, Zephyr RTOS), sistem IoT (ESP32, STM32, nRF52), nRF Connect SDK, Altium Designer, pengembangan perangkat medis, teknologi bioelektroda, integrasi perangkat keras-perangkat lunak, konsultasi teknis, R\&D, dan debugging JTAG.
\end{cvparagraph}

%-------------------------------------------------------------------------------
% SECTION: Work Experience
%-------------------------------------------------------------------------------
\cvsection{Pengalaman Profesional}

\begin{cventries}

%---------------------------------------------------------
\vspace{2pt}
\cventry
    {Insinyur Perangkat Keras Biosensor/EEG} % Job title
    {Niura (MindTune Innovations regional office, Pakistan)} % Organization
    {Wah Cantt, Pakistan} % Location
    {Mar. 2025 - Sekarang} % Date(s)
    {
      \begin{cvitems}
        \item {Merancang dan mengembangkan perangkat keras EEG serta sistem bioelektroda dengan fokus pada akuisisi biosinyal yang stabil dan berkualitas tinggi untuk penggunaan nyata}
        \item {Merekayasa perangkat keras earbud kompak dan PCB multilayer untuk mendukung kebutuhan sensor neural sekaligus keterbatasan perangkat wearable konsumen}
        \item {Mengembangkan firmware tertanam untuk akuisisi EEG real-time, filtering, serta transmisi nirkabel yang andal dari earbud mini}
        \item {Mengintegrasikan fitur audio stereo TWS agar perangkat tetap berfungsi sebagai earbud normal sambil mendukung aliran data EEG}
        \item {Meningkatkan kualitas sinyal melalui tuning perangkat keras, optimasi antarmuka elektroda, reduksi noise, dan pengujian validasi}
        \item {Melaksanakan verifikasi perangkat keras end-to-end termasuk bring-up, debugging, uji performa, dan penyempurnaan prototipe iteratif}
        \item {Berkolaborasi lintas tim perangkat keras, firmware, dan produk untuk menghasilkan desain siap produksi yang selaras dengan standar medis dan mutu}
      \end{cvitems}
    }

%---------------------------------------------------------
\vspace{2pt}
\cventry
    {Konsultan Teknis (Remote/Sessional)} % Job title
    {Zinovaa} % Organization
    {Remote} % Location
    {Mei 2025 - Sekarang} % Date(s)
    {
      \begin{cvitems}
        \item {Memberikan konsultasi ahli terkait arsitektur sistem tertanam, keputusan desain perangkat keras, dan strategi optimasi firmware}
        \item {Membimbing tim pengembangan dalam pemilihan mikrokontroler, sensor, dan komponen yang tepat untuk aplikasi IoT dan wearable}
        \item {Meninjau dan mengoptimasi skematik serta layout PCB untuk integritas sinyal, distribusi daya, dan manajemen termal}
        \item {Memberikan arahan implementasi Bluetooth Low Energy (BLE), Wi-Fi, dan LoRa pada sistem IoT nirkabel berbasis ESP32, STM32, dan nRF52}
        \item {Melakukan sesi troubleshooting jarak jauh menggunakan debugger JTAG, logic analyzer, dan osiloskop untuk menyelesaikan isu perangkat keras kompleks}
        \item {Memastikan integrasi perangkat keras-perangkat lunak yang kokoh melalui code review, panduan implementasi HAL, dan konfigurasi RTOS}
        \item {Memberikan arahan strategis terkait praktik terbaik design-for-manufacturing (DFM) dan design-for-testability (DFT)}
      \end{cvitems}
    }

%---------------------------------------------------------
\vspace{2pt}
\cventry
    {Ketua Tim Insinyur Perangkat Keras Tertanam} % Job title
    {Revive Medical Technologies} % Organization
    {Rawalpindi, Pakistan} % Location
    {Feb. 2024 - Mar. 2025} % Date(s)
    {
      \begin{cvitems}
        \item {Memimpin tim PCB dalam desain skematik, layout, fabrikasi, pengujian, dan implementasi untuk proyek perangkat keras tertanam}
        \item {Mengelola pembuatan BOM dan perencanaan komponen sambil mengawasi alur perakitan serta proses cetak PCB agar memenuhi standar kualitas}
        \item {Berkoordinasi dengan tim engineering lintas fungsi untuk memastikan milestone perangkat keras tercapai tepat waktu sesuai target proyek}
        \item {Mengembangkan dan memvalidasi prototipe perangkat keras tertanam melalui bring-up terstruktur, debugging, dan pengujian verifikasi}
        \item {Membangun simulasi otomasi industri di Factory I/O dan mengintegrasikan logika PLC untuk skenario kontrol proses dunia nyata}
        \item {Berkontribusi pada peningkatan proses melalui design review, dokumentasi teknis, dan dukungan troubleshooting lintas tim}
        \item {Mendukung proses serah-terima ke manufaktur dengan memvalidasi file desain, checklist pengujian, dan dokumentasi produksi agar build dapat diulang secara konsisten}
      \end{cvitems}
    }

%---------------------------------------------------------

\end{cventries}

%-------------------------------------------------------------------------------
% SECTION: Education
%-------------------------------------------------------------------------------
\clearpage
\cvsection{Pendidikan}

\begin{cventries}

%---------------------------------------------------------
\vspace{2pt}
\cventry
    {Sarjana Sains Teknik Elektro (Konsentrasi Komputer)} % Degree
    {COMSATS University Islamabad} % Institution
    {Islamabad, Pakistan} % Location
    {Aug. 2019 - Feb. 2024} % Date(s)
    {
      \begin{cvitems}
        \item {Jurusan: Teknik Elektro dengan spesialisasi Sistem Tertanam, Pemrosesan Sinyal Digital, dan Elektronika Daya}
        \item {Mata kuliah relevan: Sistem Mikroprosesor, Desain FPGA, Sistem Kendali, Pengolahan Citra Digital, Komunikasi Data \& Jaringan Komputer, Otomasi Industri, dan Desain Rangkaian}
        \item {Mengembangkan keahlian MATLAB, NI LabVIEW, Proteus, dan Altium Designer melalui proyek akademik dan praktikum laboratorium}
        \item {Proyek tugas akhir: Smart Door Lock System berbasis ESP32-CAM dengan pemrosesan citra teroptimasi dan autentikasi nirkabel}
        \item {Memperoleh pengalaman langsung menggunakan board pengembangan ARM Cortex-M, ESP32, STM32, dan FPGA}
      \end{cvitems}
    }

%---------------------------------------------------------
\vspace{2pt}
\cventry
    {Higher Secondary School Certificate - Pra-Teknik} % Degree
    {University College for Boys, Univ. of Peshawar} % Institution
    {Peshawar, Pakistan} % Location
    {2017 - 2019} % Date(s)
    {
      \begin{cvitems}
        \item {Mempelajari Matematika, Fisika, dan Kimia dengan fokus pada kemampuan analitis dan pemecahan masalah}
        \item {Membangun fondasi kuat pada prinsip keteknikan, kalkulus, mekanika, dan metodologi ilmiah}
      \end{cvitems}
    }

%---------------------------------------------------------
\vspace{2pt}
\cventry
    {Secondary School Certificate - Kelompok Sains} % Degree
    {Islamia Collegiate School} % Institution
    {Peshawar, Pakistan} % Location
    {2015 - 2017} % Date(s)
    {
      \begin{cvitems}
        \item {Menyelesaikan studi dasar Matematika, Fisika, Kimia, dan Biologi}
        \item {Mengembangkan performa akademik yang kuat serta kemampuan berpikir kritis}
      \end{cvitems}
    }

\end{cventries}

%-------------------------------------------------------------------------------
% SECTION: Technical Skills
%-------------------------------------------------------------------------------
\cvsection{Keahlian Teknis}

\begin{cvskills}

%---------------------------------------------------------
\cvskill
    {Sistem Tertanam} 
    {ESP32, STM32 (Cortex-M3/M4), nRF52 Series, AVR, i.MX8MM MPU, ARM Cortex Architecture, Bare-Metal \& RTOS Programming}

%---------------------------------------------------------
\cvskill
    {Pengembangan Firmware} 
    {C, C++, Embedded C, FreeRTOS, Zephyr RTOS, nRF Connect SDK, HAL/LL Drivers, Bootloader Development, DFU/OTA Updates}

%---------------------------------------------------------
\cvskill
    {Desain PCB \& Alat CAD} 
    {Altium Designer, KiCad, Autodesk Eagle, PADS, Proteus, OrCAD, High-Speed \& Multilayer PCB Design (up to 12 layers)}

%---------------------------------------------------------
\cvskill
    {Integritas Sinyal \& Daya} 
    {Impedance Control, Controlled Routing, Ground Plane Optimization, PDN Analysis, SI/PI Simulation, EMI/EMC Compliance}

%---------------------------------------------------------
\cvskill
    {Biosensor \& Perangkat Medis} 
    {EEG Hardware Design (ADS1299), ECG/PPG Acquisition (MAX86150), Bioelectrode Technology, AFE Design, IEC 60601-1 Compliance}

%---------------------------------------------------------
\cvskill
    {Protokol Komunikasi} 
    {I2C, SPI, UART, USB 2.0/3.0, CAN Bus, BLE 5.x, Wi-Fi (802.11), LoRa, MQTT, WebSocket, Modbus}

%---------------------------------------------------------
\cvskill
    {Sistem Nirkabel \& IoT} 
    {Bluetooth Low Energy (BLE), Nordic nRF52, ESP32 Wi-Fi/BLE, LoRa/LoRaWAN, Wireless Charging (Qi Standard, BQ51003)}

%---------------------------------------------------------
\cvskill
    {Otomasi Industri} 
    {Siemens S7-300/S7-1200 PLC, TIA Portal, Factory I/O, Ladder Logic, SCADA Systems, Process Control}

%---------------------------------------------------------
\cvskill
    {Manufaktur \& Pengujian} 
    {CNC Programming (G-Code), PCB Fabrication, Pick-and-Place Assembly, BOM Management, DFM/DFT, IPC-7351 Standards}

%---------------------------------------------------------
\cvskill
    {Alat Debugging Perangkat Keras} 
    {JTAG/SWD Debuggers, Oscilloscopes (up to 500MHz), Logic Analyzers, Spectrum Analyzers, Thermal Cameras, Multimeters}

%---------------------------------------------------------
\cvskill
    {Perangkat Lunak \& Simulasi} 
    {MATLAB, Simulink, NI LabVIEW, LTspice, ANSYS HFSS, Python (NumPy, SciPy, Matplotlib), Git/GitHub}

%---------------------------------------------------------
\cvskill
    {Standar Desain \& Kepatuhan} 
    {IPC-2221, IPC-7351, IEC 60601-1, ISO 13485, RoHS 3, CE Marking, FDA Class II Device Requirements}

%---------------------------------------------------------
\cvskill
    {Manajemen Proyek} 
    {Agile/Scrum Methodologies, Jira, Confluence, Bitbucket, Cross-functional Team Leadership, Technical Documentation}

%---------------------------------------------------------
\cvskill
    {Bahasa} 
    {Inggris (Profesional), Urdu (Penutur Asli)}

\end{cvskills}

%-------------------------------------------------------------------------------
% SECTION: Key Projects
%-------------------------------------------------------------------------------
\cvsection{Proyek Profesional Utama}

\begin{cvparagraph}

\textbf{1. Smartwatch Generasi Baru dengan Pengisian Daya Nirkabel}
\begin{cvprojitems}
    \item Merancang arsitektur perangkat keras lengkap untuk wearable canggih berbasis nRF5340 dual-core SoC (ARM Cortex-M33).
    \item Mengimplementasikan receiver pengisian daya nirkabel BQ51003 dengan PMIC ADP5360 untuk manajemen daya yang optimal.
    \item Mengintegrasikan MAX86150 untuk pengukuran ECG, PPG, dan SpO\textsubscript{2} dengan akurasi tingkat klinis.
    \item Mengembangkan sistem layar ganda: IPS LCD 1.54" untuk UI interaktif + E-Ink berdaya rendah untuk tampilan always-on.
    \item Menambahkan sensor lingkungan BME680 (suhu, kelembapan, tekanan) dan CCS811 (monitoring kualitas udara).
    \item Mendesain integrasi IMU ICM-20689 6-axis untuk pelacakan gerak dan pengenalan gestur.
    \item Mencapai konsumsi arus rata-rata <30mA dengan ketahanan baterai 7 hari pada baterai LiPo 300mAh.
    \item Mengimplementasikan stack Nordic BLE 5.2 dengan DFU over-the-air yang aman.
\end{cvprojitems}

\textbf{2. Earbud True Wireless Stereo (TWS) dengan Jieli IC}
\begin{cvprojitems}
    \item Mengembangkan desain PCB lengkap untuk earbud TWS menggunakan audio SoC Jieli Bluetooth 5.0.
    \item Mengimplementasikan pemrosesan audio lanjutan dengan ANC (Active Noise Cancellation) dan ENC (Environmental Noise Cancellation).
    \item Mendesain charging case kompak dengan kemampuan pengisian nirkabel dan sistem manajemen baterai.
    \item Mencapai latensi audio rendah <60ms untuk aplikasi gaming dan video.
    \item Mengoptimasi desain antena untuk konektivitas Bluetooth yang stabil hingga jarak 10 meter.
\end{cvprojitems}

\textbf{3. Smart Earbuds dengan EEG Sensing}
\begin{cvprojitems}
    \item Mempelopori desain inovatif yang mengintegrasikan sensor EEG dry-electrode ke dalam form factor earbud TWS.
    \item Mengembangkan rangkaian akuisisi biopotensial dengan noise floor <1\ensuremath{\mu}V untuk deteksi sinyal neural in-ear.
    \item Mengimplementasikan algoritma analisis mood/emosi real-time berbasis machine learning pada MCU tertanam.
    \item Membuat sistem penyesuaian audio dinamis yang merespons kondisi emosi terdeteksi.
    \item Mendesain PCB mini dengan isolasi RF antara Bluetooth dan frontend EEG analog.
\end{cvprojitems}

\textbf{4. Perangkat Keras Medis (SDCM-II, DermScope, UV Curing Chamber)}
\begin{cvprojitems}
    \item \textit{Stent Drug Coating Machine II (SDCM-II)}: Merancang elektronika kontrol gerak presisi dengan driver stepper motor dan closed-loop feedback untuk pelapisan farmasi otomatis.
    \item \textit{DermScope}: Mengembangkan perangkat pencitraan medis kompak dengan antarmuka kamera resolusi tinggi, kontrol pencahayaan LED, dan transfer data USB 2.0.
    \item \textit{UV Curing Chamber}: Merekayasa sistem kontrol tertanam dengan regulasi suhu presisi, monitoring intensitas UV, dan manajemen siklus curing otomatis.
    \item Seluruh proyek dirancang dengan isolasi keselamatan IEC 60601-1 dan komponen kelas medis.
\end{cvprojitems}

\textbf{5. Smartwatch NRF Kompak 49\,$\times$\,49 mm}
\begin{cvprojitems}
    \item Mendesain PCB smartwatch ultra-kompak berbasis nRF52832 SoC dan layar OLED 0.96".
    \item Mengimplementasikan teknik miniaturisasi agresif termasuk via-in-pad dan teknologi HDI.
    \item Mencapai ketebalan total PCB <2mm termasuk perakitan baterai dan layar.
    \item Mengoptimasi konsumsi daya: <50\ensuremath{\mu}A saat deep sleep dan <5mA saat koneksi BLE aktif.
\end{cvprojitems}

\textbf{6. Perangkat Wearable Mouthguard Atletik}
\begin{cvprojitems}
    \item Mengembangkan wearable berbasis NRF yang ditanam pada mouthguard atletik untuk pemantauan kesehatan real-time.
    \item Mengintegrasikan sensor suhu, SpO\textsubscript{2}, gaya gigitan (FSR), dan akselerometer 3-axis untuk deteksi benturan.
    \item Mendesain enclosure tahan air dan tahan benturan dengan material biokompatibel.
    \item Mengimplementasikan komunikasi BLE untuk streaming data real-time ke aplikasi mobile.
    \item Membuat charging dock kustom dengan antarmuka pogo-pin untuk pengisian ulang baterai.
\end{cvprojitems}

\textbf{7. Sistem Pemrosesan Data EEG OpenBCI}
\begin{cvprojitems}
    \item Mengembangkan firmware kustom untuk board OpenBCI berbasis ADS1299 dengan akuisisi EEG 8 kanal.
    \item Mengimplementasikan protokol komunikasi SPI dengan laju sampling 16kSPS pada seluruh kanal.
    \item Membuat pipeline pemrosesan data berbasis Python untuk analisis gelombang otak real-time (alpha, beta, theta, delta).
    \item Mengintegrasikan algoritma machine learning untuk deteksi artefak dan klasifikasi sinyal.
\end{cvprojitems}

\textbf{8. Bootloader Kustom DFU/OTA untuk nRF52840}
\begin{cvprojitems}
    \item Mengembangkan bootloader aman untuk Adafruit nRF52840 Sense agar mendukung pembaruan firmware over-the-air.
    \item Mengimplementasikan verifikasi tanda tangan kriptografis menggunakan library mbedTLS.
    \item Membuat manajemen memori flash dual-bank untuk mekanisme pembaruan fail-safe.
    \item Mencapai waktu pembaruan <30 detik untuk image firmware 512KB.
\end{cvprojitems}

\end{cvparagraph}

%-------------------------------------------------------------------------------
% SECTION: Final Year Project
%-------------------------------------------------------------------------------
\cvsection{Proyek Tugas Akhir}

\begin{cventries}

\vspace{2pt}
\cventry
    {Ketua Tim \& Perancang Perangkat Keras} % Role
    {Sistem Kunci Pintu Pintar} % Project Title
    {COMSATS University Islamabad} % Institution
    {Aug. 2022 - Jul. 2023} % Date(s)
    {
      \begin{cvitems}
        \item {Mengembangkan sistem kunci pintu pintar terintegrasi menggunakan ESP32-CAM dan ESP32 untuk kontrol akses dan pemantauan nirkabel}
        \item {Mengimplementasikan library TJpeg untuk streaming video real-time 25 FPS dengan kompresi citra yang teroptimasi}
        \item {Mendesain mekanisme penguncian modular yang kompatibel dengan pintu eksisting menggunakan solenoid lock dan aktuasi berbasis servo}
        \item {Mengintegrasikan komunikasi audio dua arah menggunakan mikrofon digital I2S dan rangkaian amplifier}
        \item {Mengembangkan sistem autentikasi kustom yang mendukung RFID, sidik jari, dan pembukaan berbasis aplikasi mobile}
        \item {Membuat antarmuka pengguna intuitif pada TFT LCD 3.5" yang menampilkan video langsung dan informasi pengunjung}
        \item {Mengimplementasikan mode daya rendah dengan arus standby <100mA dan kemampuan wake-up instan}
        \item {Mendesain PCB dengan catu daya terintegrasi (5V/2A), modul ESP32, dan antarmuka periferal}
        \item {Berhasil mendemonstrasikan sistem dengan waktu respons autentikasi <1 detik dan akurasi pengenalan 99\%}
      \end{cvitems}
    }

\end{cventries}

%-------------------------------------------------------------------------------
% SECTION: Open Source Contributions
%-------------------------------------------------------------------------------
\cvsection{Kontribusi Open Source}

\begin{cventries}

\vspace{2pt}
\cventry
    {Maintainer \& Pembuat} % Role
    {Library Komponen Altium Kustom} % Project Title
    {GitHub - Repositori Publik} % Platform
    {Berjalan} % Date(s)
    {
      \begin{cvitems}
        \item {Mengembangkan library komponen Altium Designer komprehensif berisi 500+ komponen yang dioptimasi untuk desain PCB cepat}
        \item {Setiap komponen disimpan pada file terpisah dengan struktur modular untuk integrasi mudah dan version control}
        \item {Memastikan kepatuhan 100\% terhadap RoHS 3 dengan dokumentasi lingkungan dan keselamatan yang rinci}
        \item {Mendesain footprint sesuai standar IPC-7351B dengan toleransi manufaktur yang diverifikasi melalui DRC}
        \item {Memprioritaskan kompatibilitas dengan JLCPCB Parts Library untuk fabrikasi dan pengadaan komponen yang lancar}
        \item {Menyertakan model 3D untuk verifikasi jarak mekanik yang akurat pada Altium 3D viewer}
        \item {Dikelola aktif dengan pembaruan dua mingguan untuk menambah komponen terbaru dan memperbaiki isu yang dilaporkan}
        \item {Mendapatkan 100+ stars serta kontribusi dari komunitas sistem tertanam di berbagai negara}
      \end{cvitems}
    }

\end{cventries}

%-------------------------------------------------------------------------------
% SECTION: Certifications
%-------------------------------------------------------------------------------
\cvsection{Sertifikasi Profesional}

\begin{cvhonors}

\cvhonor
    {Insinyur Elektro Terdaftar PEC} % Title
    {Nomor Registrasi: ELECT/107509} % Details
    {Pakistan Engineering Council (PEC)} % Issuer
    {2024} % Year

\cvhonor
    {Certified PCB Designer} % Title
    {Penyelesaian Altium Education PCB Basic Design Course (Kredensial: cert\_c8lmf4pd)} % Details
    {Altium} % Issuer
    {Sep. 2023} % Year

\cvhonor
    {Learn Altium Essentials -- Second Edition} % Title
    {Teknik Desain PCB Lanjutan dan Praktik Terbaik} % Details
    {Altium} % Issuer
    {2024} % Year

\cvhonor
    {Desain PCB High-Speed \& Multi-layer} % Title
    {Teknik lanjutan untuk integritas sinyal dan kontrol impedansi} % Details
    {Sertifikasi Online} % Issuer
    {2024} % Year

\cvhonor
    {Spesialis Integritas Sinyal \& Daya} % Title
    {Kursus Lanjutan Analisis SI/PI untuk Perancang PCB} % Details
    {Sintecs Training Program} % Issuer
    {2024} % Year

\cvhonor
    {Dasar-dasar nRF Connect SDK} % Title
    {Pelatihan praktis pada platform IoT milik Nordic Semiconductor} % Details
    {Nordic Semiconductor Academy} % Issuer
    {2025} % Year

\cvhonor
    {Pengantar IoT} % Title
    {Dasar-dasar arsitektur dan aplikasi Internet of Things} % Details
    {Cisco Networking Academy} % Issuer
    {Mar. 2024} % Year

\end{cvhonors}

%-------------------------------------------------------------------------------
% SECTION: Key Achievements
%-------------------------------------------------------------------------------
\clearpage
\cvsection{Pencapaian Utama \& Pengakuan}

\begin{cvhonors}

\cvhonor
    {Kepemimpinan Tim} % Category
    {Memimpin tim lintas fungsi beranggotakan 5 engineer untuk menyelesaikan 10+ proyek perangkat medis tepat waktu dan sesuai anggaran} % Achievement
    {} % Organization
    {2024-2025} % Year

\cvhonor
    {Peningkatan Proses} % Category
    {Menurunkan cacat manufaktur PCB sebesar 40\% melalui penerapan standar DFM/DFT dan design review otomatis} % Achievement
    {} % Organization
    {2024} % Year

\cvhonor
    {Keberhasilan Freelance} % Category
    {Menyelesaikan 15+ proyek Upwork dengan tingkat kepuasan klien 100\% dan rata-rata ulasan 5.0/5.0} % Achievement
    {} % Organization
    {2024-2025} % Year

\cvhonor
    {Dampak Open Source} % Category
    {Library komponen Altium memperoleh 100+ GitHub stars dan diadopsi oleh developer di 15+ negara} % Achievement
    {} % Organization
    {2024-2025} % Year

\cvhonor
    {Inovasi Teknis} % Category
    {Mempelopori integrasi sensor EEG ke form factor earbud TWS untuk audio responsif terhadap emosi} % Achievement
    {} % Organization
    {2025} % Year

\end{cvhonors}

%-------------------------------------------------------------------------------
\end{document}
